\documentclass[a4paper]{article}

\usepackage{tcolorbox}
\usepackage[utf8]{inputenc}
\usepackage[T1]{fontenc}
\usepackage{textcomp}
\usepackage{amsmath, amssymb, amsthm}
\tcbuselibrary{theorems}
\newtcbtheorem
  []% init options
  {definition}% name
  {Definition}% title
  {%
    colback=blue!5,
    colframe=blue!35!black,
    fonttitle=\bfseries,
  }% options
  {def}% prefix
\usepackage[english]{babel}
\newcommand{\R}{\mathbb{R}}
\newcommand{\N}{\mathbb{N}}
\newcommand{\Q}{\mathbb{Q}}
\newcommand{\C}{\mathbb{C}}
\newcommand{\Z}{\mathbb{Z}}
\newcommand{\cat}{%
  \textsf%
}

\newtheorem{theorem}{Theorem}
\theoremstyle{definition}
\newtheorem{example}{Example}
\newtheorem{proposition}{Proposition}

\setlength{\parindent}{0pt}

\title{Test template}
\author{Zarak}
\date{\today}

\begin{document}
\maketitle

\begin{definition}{Group}{def:group}
  A \textit{group} is a groupoid with a single object.
\end{definition}

\begin{definition}{Initial and final objects}{def:initial_final_objects}
  Let $\cat{C}$ be a category. We say that an object $I$ of $\cat{C}$ is
  \textit{initial} in $\cat{C}$ if for every object $A$ of $\cat{C}$ there exists
  \textit{exactly one} morphism $I \to  A$ in $\cat{C}$:

  \begin{equation*}
    \forall A \in \cat{Obj(C)} : \quad \cat{Hom}_\cat{C}(I, A) \text{ is
    a singleton }
  \end{equation*}
\end{definition}

\begin{proposition}
  Let $\cat{C}$ be a category.
  \begin{enumerate}
    \item If $ I_1, I_2$ are both initial objects in $\cat{C}$, then $ I_1
      \cong I_2$
    \item If $F_1, F_2$ are both final objects in $ \cat{C}$, then $ F_1 \cong
      F_2$.
  \end{enumerate}
\end{proposition}

\end{document}
